\documentclass[12pt,a4paper]{article}

% Polskie znaki i formatowanie
\usepackage[utf8]{inputenc}
\usepackage[polish]{babel}
\usepackage[T1]{fontenc}

% Pakiety graficzne i tabele
\usepackage{graphicx}
\usepackage{float}
\usepackage{booktabs}
\usepackage{multirow}
\usepackage{longtable}
\usepackage{array}

% Kolory i linki
\usepackage{xcolor}
\usepackage{hyperref}
\hypersetup{
    colorlinks=true,
    linkcolor=blue,
    filecolor=magenta,
    urlcolor=cyan,
    citecolor=green,
}

% Matematyka i kod
\usepackage{amsmath}
\usepackage{amssymb}
\usepackage{listings}
\usepackage{courier}

% Ustawienia listingów (kod)
\lstset{
    basicstyle=\ttfamily\small,
    breaklines=true,
    frame=single,
    numbers=left,
    numberstyle=\tiny,
    tabsize=4,
    showstringspaces=false
}

% Marginesy
\usepackage[left=2.5cm, right=2.5cm, top=3cm, bottom=3cm]{geometry}

% Nagłówki i stopki
\usepackage{fancyhdr}
\pagestyle{fancy}
\fancyhf{}
\rhead{\thepage}
\lhead{Projekt: Implementacja i analiza sieci neuronowych MLP}

% Metadane dokumentu
\title{
    \Huge{\textbf{Projekt: Implementacja i analiza\\sieci neuronowych MLP}} \\
    \vspace{0.5cm}
    \Large{Porównanie implementacji ręcznej i frameworkowej}
}

\author{
    Jakub Sornat \\
    Maciej Tajs \\
    Bartłomiej Sadza
}

\date{\today}

\begin{document}

\maketitle
\thispagestyle{empty}

\newpage
\tableofcontents
\newpage

% ============================================================================
\section{Wstęp}
% ============================================================================

\subsection{Cel projektu}

Celem projektu jest implementacja wielowarstwowego perceptronu (MLP, Multi-Layer Perceptron)
w dwóch wariantach:
\begin{itemize}
    \item \textbf{Implementacja ręczna} -- napisana od podstaw w NumPy, z pełną kontrolą nad
          forward propagation, backward propagation i optymalizacją
    \item \textbf{Implementacja frameworkowa} -- wykorzystująca bibliotekę TensorFlow/Keras
\end{itemize}

Projekt ma na celu:
\begin{enumerate}
    \item Zrozumienie wewnętrznego działania sieci neuronowych poprzez implementację od podstaw
    \item Porównanie wydajności i dokładności obu podejść
    \item Praktyczne zastosowanie MLP do problemów klasyfikacji i regresji
    \item Przeprowadzenie systematycznych eksperymentów z różnymi hiperparametrami
\end{enumerate}

\subsection{Zakres prac}

W ramach projektu zrealizowano:
\begin{itemize}
    \item Implementację ręczną sieci MLP z warstwami Dense, aktywacjami ReLU i Softmax
    \item Implementację modelu MLP w Keras/TensorFlow
    \item Preprocessing i przygotowanie 4 zbiorów danych (2 klasyfikacyjne, 2 regresyjne)
    \item Przeprowadzenie eksperymentów z grid search po hiperparametrach
    \item Analizę wyników i wizualizację (learning curves, confusion matrices, wykresy porównawcze)
    \item Dokumentację techniczną projektu
\end{itemize}

\subsection{Użyte technologie}

\begin{itemize}
    \item \textbf{Python 3.x} -- główny język programowania
    \item \textbf{NumPy} -- operacje na macierzach i implementacja ręczna
    \item \textbf{Pandas} -- przetwarzanie danych
    \item \textbf{TensorFlow/Keras} -- implementacja frameworkowa
    \item \textbf{Matplotlib} -- wizualizacja wyników
    \item \textbf{scikit-learn} -- metryki ewaluacji i preprocessing
    \item \textbf{OpenPyXL} -- eksport wyników do Excel
\end{itemize}

% ============================================================================
\section{Zbiory danych}
% ============================================================================

W projekcie wykorzystano cztery zbiory danych: dwa do klasyfikacji binarnej
i dwa do regresji.

\subsection{Classification: Adult Income Dataset}

\textbf{Źródło:} UCI Machine Learning Repository

\textbf{Opis:} Zbiór danych zawierający informacje o dochodach osób dorosłych.
Celem jest przewidzenie, czy osoba zarabia powyżej 50K\$ rocznie.

\textbf{Charakterystyka:}
\begin{itemize}
    \item Liczba rekordów: 32,561
    \item Liczba cech (po preprocessingu): 21
    \item Klasy: 2 (>50K, ≤50K)
    \item Typ problemu: Klasyfikacja binarna
\end{itemize}

\textbf{Przykładowe cechy:}
\begin{itemize}
    \item Wiek (age)
    \item Wykształcenie (education)
    \item Stan cywilny (marital\_status)
    \item Zawód (occupation)
    \item Godziny pracy tygodniowo (hours\_per\_week)
\end{itemize}

\subsection{Classification (Our): Loan Approval Dataset}

\textbf{Źródło:} [Podać źródło jeśli znane]

\textbf{Opis:} Zbiór danych dotyczący zatwierdzania kredytów bankowych.

\textbf{Charakterystyka:}
\begin{itemize}
    \item Liczba rekordów: 50,000
    \item Liczba cech (po preprocessingu): 9
    \item Klasy: 2 (Approved, Rejected)
    \item Typ problemu: Klasyfikacja binarna
\end{itemize}

\textbf{Przykładowe cechy:}
\begin{itemize}
    \item Dochód roczny (annual\_income)
    \item Wynik kredytowy (credit\_score)
    \item Liczba niespłaconych kredytów (defaults\_on\_file)
\end{itemize}

\subsection{Regression: Stock Market Dataset}

\textbf{Źródło:} [Podać źródło jeśli znane]

\textbf{Opis:} Dane z rynku akcji zawierające szeregi czasowe cen i wskaźników.

\textbf{Charakterystyka:}
\begin{itemize}
    \item Liczba rekordów: 29,900
    \item Liczba cech: [liczba po preprocessingu]
    \item Zmienna docelowa: Cena zamknięcia (closing price)
    \item Typ problemu: Regresja
\end{itemize}

\subsection{Regression (Our): Student Performance Dataset}

\textbf{Źródło:} [Podać źródło jeśli znane]

\textbf{Opis:} Zbiór danych o wynikach egzaminów uczniów.

\textbf{Charakterystyka:}
\begin{itemize}
    \item Liczba rekordów: 6,607
    \item Liczba cech (po preprocessingu): 5
    \item Zmienna docelowa: Wynik egzaminu (exam score)
    \item Typ problemu: Regresja
\end{itemize}

\textbf{Cechy:}
\begin{itemize}
    \item Godziny nauki (study\_hours)
    \item Zaangażowanie rodziców (parental\_involvement)
    \item Dostęp do zasobów edukacyjnych (access\_to\_resources)
\end{itemize}

% ============================================================================
\section{Preprocessing danych}
% ============================================================================

\subsection{Ogólna procedura}

Dla każdego zbioru danych przeprowadzono następujące kroki preprocessingu:

\begin{enumerate}
    \item \textbf{Obsługa brakujących wartości} -- usuwanie lub imputacja
    \item \textbf{Kodowanie zmiennych kategorycznych}
    \begin{itemize}
        \item One-hot encoding dla zmiennych nominalnych
        \item Target encoding dla zmiennych o wysokiej kardynalności
        \item Binary encoding dla zmiennych binarnych
        \item Ordinal encoding dla zmiennych porządkowych
    \end{itemize}
    \item \textbf{Feature engineering} -- tworzenie nowych cech (np. logarytmy, ilorazy)
    \item \textbf{Selekcja cech} -- analiza korelacji, usuwanie cech o niskiej korelacji z targetem
    \item \textbf{Balansowanie klas} (dla klasyfikacji) -- undersampling klasy większościowej
    \item \textbf{Standaryzacja} -- normalizacja cech numerycznych (mean=0, std=1)
    \item \textbf{Podział danych} -- train/test (80/20) oraz train/val/test (70/15/15)
\end{enumerate}

\subsection{Adult Income Dataset}

\textbf{Specyficzne operacje:}
\begin{itemize}
    \item Usunięto brakujące wartości w kolumnach: workclass, occupation, native\_country
    \item Binary encoding: sex, native\_country
    \item Target encoding: occupation (ze względu na wysoką kardynalność)
    \item One-hot encoding: marital\_status, relationship, workclass
    \item Feature engineering: capital\_net\_log = log(capital\_gain - capital\_loss + 1)
    \item Undersampling: zbalansowano klasy do proporcji 1:1
\end{itemize}

\subsection{Loan Approval Dataset}

\textbf{Specyficzne operacje:}
\begin{itemize}
    \item Selekcja cech: usunięto cechy o korelacji < 0.15 z targetem
    \item Transformacja logarytmiczna: annual\_income, defaults\_on\_file
    \item Undersampling: zbalansowano klasy
    \item Standaryzacja wszystkich cech numerycznych
\end{itemize}

\subsection{Stock Market Dataset}

\textbf{Specyficzne operacje:}
\begin{itemize}
    \item Zachowanie uporządkowania czasowego (time-based split)
    \item Standaryzacja wszystkich wskaźników
    \item Podział czasowy: 80/20 i 70/15/15
\end{itemize}

\subsection{Student Performance Dataset}

\textbf{Specyficzne operacje:}
\begin{itemize}
    \item Ordinal encoding: Parental\_Involvement, Access\_to\_Resources
    \item Binary encoding zmiennych kategorycznych
    \item Selekcja cech: |r| > 0.15 z targetem
    \item Standaryzacja tylko cech numerycznych (zachowanie interpretacji cech ordinalnych)
\end{itemize}

% ============================================================================
\section{Architektura sieci neuronowych}
% ============================================================================

\subsection{Architektura ogólna}

Obie implementacje (ręczna i Keras) wykorzystują tę samą architekturę MLP:

\begin{itemize}
    \item \textbf{Warstwa wejściowa:} liczba neuronów = liczba cech
    \item \textbf{Warstwy ukryte:} N warstw Dense z aktywacją ReLU
    \item \textbf{Warstwa wyjściowa:}
    \begin{itemize}
        \item Klasyfikacja: Dense(n\_classes) + Softmax
        \item Regresja: Dense(1) + aktywacja liniowa
    \end{itemize}
\end{itemize}

\subsection{Implementacja ręczna}

\subsubsection{Warstwa Dense}

Warstwa Dense implementuje transformację liniową:
\begin{equation}
    \mathbf{y} = \mathbf{W}^T \mathbf{x} + \mathbf{b}
\end{equation}

\textbf{Inicjalizacja wag:} Xavier/Glorot uniform
\begin{equation}
    W \sim \mathcal{U}\left(-\sqrt{\frac{6}{n_{in} + n_{out}}}, \sqrt{\frac{6}{n_{in} + n_{out}}}\right)
\end{equation}

\textbf{Backward propagation:}
\begin{align}
    \frac{\partial L}{\partial \mathbf{W}} &= \mathbf{x}^T \frac{\partial L}{\partial \mathbf{y}} \\
    \frac{\partial L}{\partial \mathbf{b}} &= \sum \frac{\partial L}{\partial \mathbf{y}} \\
    \frac{\partial L}{\partial \mathbf{x}} &= \frac{\partial L}{\partial \mathbf{y}} \mathbf{W}^T
\end{align}

\subsubsection{Aktywacja ReLU}

\begin{equation}
    \text{ReLU}(x) = \max(0, x)
\end{equation}

Gradient:
\begin{equation}
    \frac{\partial \text{ReLU}(x)}{\partial x} = \begin{cases}
        1 & \text{jeśli } x > 0 \\
        0 & \text{w przeciwnym razie}
    \end{cases}
\end{equation}

\subsubsection{Aktywacja Softmax (klasyfikacja)}

\begin{equation}
    \text{Softmax}(\mathbf{z})_i = \frac{e^{z_i}}{\sum_{j=1}^{K} e^{z_j}}
\end{equation}

\subsubsection{Funkcja straty}

\textbf{Klasyfikacja:} Cross-entropy loss
\begin{equation}
    L = -\frac{1}{N} \sum_{i=1}^{N} \log(\hat{y}_{i,c_i})
\end{equation}
gdzie $c_i$ to prawdziwa klasa dla próbki $i$.

\textbf{Regresja:} Mean Squared Error (MSE)
\begin{equation}
    L = \frac{1}{N} \sum_{i=1}^{N} (y_i - \hat{y}_i)^2
\end{equation}

\subsubsection{Optymalizator}

Stochastic Gradient Descent (SGD):
\begin{equation}
    W \leftarrow W - \eta \frac{\partial L}{\partial W}
\end{equation}
gdzie $\eta$ to learning rate.

\subsection{Implementacja Keras}

Implementacja Keras wykorzystuje wysokopoziomowe API:

\begin{lstlisting}[language=Python, caption=Przykładowa architektura w Keras]
model = keras.Sequential()
model.add(layers.Input(shape=(n_inputs,)))

# Warstwy ukryte
for i in range(n_hidden_layers):
    model.add(layers.Dense(
        n_neurons,
        activation='relu',
        kernel_initializer='glorot_uniform'
    ))

# Warstwa wyjściowa
if task_type == "classification":
    model.add(layers.Dense(n_outputs, activation='softmax'))
    model.compile(
        optimizer=optimizers.SGD(learning_rate=lr),
        loss='sparse_categorical_crossentropy',
        metrics=['accuracy']
    )
else:
    model.add(layers.Dense(1, activation='linear'))
    model.compile(
        optimizer=optimizers.SGD(learning_rate=lr),
        loss='mse',
        metrics=['mae']
    )
\end{lstlisting}

% ============================================================================
\section{Metodologia eksperymentów}
% ============================================================================

\subsection{Grid Search}

Przeprowadzono systematyczne przeszukiwanie przestrzeni hiperparametrów (grid search):

\begin{table}[H]
\centering
\begin{tabular}{ll}
\toprule
\textbf{Hiperparametr} & \textbf{Wartości} \\
\midrule
Liczba warstw ukrytych & \{1, 2, 3, 4\} \\
Liczba neuronów na warstwę & \{8, 16, 32, 64\} \\
Learning rate & \{0.001, 0.005, 0.01, 0.02\} \\
Batch size & 32 (stałe) \\
Liczba epok & 50 (stałe) \\
\midrule
\textbf{Liczba kombinacji} & 64 (4 × 4 × 4) \\
\bottomrule
\end{tabular}
\caption{Przestrzeń hiperparametrów}
\end{table}

\subsection{Procedura eksperymentu}

Dla każdej kombinacji hiperparametrów:
\begin{enumerate}
    \item Przeprowadzono \textbf{3 powtórzenia} (runs) z różną inicjalizacją
    \item Wybrano \textbf{najlepszy run} na podstawie metryki na zbiorze treningowym
    \item Zapisano wyniki (metryki train, validation, test)
\end{enumerate}

Łącznie: $64 \times 3 = 192$ treningi dla każdego zbioru danych i typu podziału.

\subsection{Metryki ewaluacji}

\subsubsection{Klasyfikacja}

\begin{itemize}
    \item \textbf{Accuracy:} $\text{Acc} = \frac{TP + TN}{TP + TN + FP + FN}$
    \item \textbf{Precision:} $\text{Prec} = \frac{TP}{TP + FP}$
    \item \textbf{Recall:} $\text{Rec} = \frac{TP}{TP + FN}$
    \item \textbf{F1-score:} $F_1 = 2 \cdot \frac{\text{Prec} \cdot \text{Rec}}{\text{Prec} + \text{Rec}}$
    \item \textbf{Cross-entropy loss}
\end{itemize}

\subsubsection{Regresja}

\begin{itemize}
    \item \textbf{MSE:} $\text{MSE} = \frac{1}{N} \sum_{i=1}^{N} (y_i - \hat{y}_i)^2$
    \item \textbf{MAE:} $\text{MAE} = \frac{1}{N} \sum_{i=1}^{N} |y_i - \hat{y}_i|$
    \item \textbf{R² score:} $R^2 = 1 - \frac{\sum (y_i - \hat{y}_i)^2}{\sum (y_i - \bar{y})^2}$
\end{itemize}

\subsection{Podziały danych}

Dla każdego zbioru zastosowano dwa warianty podziału:

\begin{enumerate}
    \item \textbf{Train/Test (80/20):}
    \begin{itemize}
        \item 80\% danych do treningu
        \item 20\% danych do testowania
    \end{itemize}

    \item \textbf{Train/Validation/Test (70/15/15):}
    \begin{itemize}
        \item 70\% danych do treningu
        \item 15\% danych do walidacji (wybór najlepszego modelu)
        \item 15\% danych do testowania końcowego
    \end{itemize}
\end{enumerate}

% ============================================================================
% WYNIKI EKSPERYMENTÓW - AUTOMATYCZNIE WYGENEROWANE
% ============================================================================

% =============================================================================
% AUTOMATYCZNIE WYGENEROWANE WYNIKI - DANE Z EKSPERYMENTÓW
% Data generacji: 2024-11-16
% =============================================================================

\section{Wyniki eksperymentów}

% =============================================================================
\subsection{Adult Income (Klasyfikacja binarna)}
% =============================================================================

\subsubsection{Najlepsze konfiguracje}

\paragraph{Manual MLP}

\begin{table}[H]
\centering
\begin{tabular}{ll}
\toprule
\textbf{Hiperparametr} & \textbf{Wartość} \\
\midrule
Liczba warstw ukrytych & 4 \\
Liczba neuronów & 16 \\
Learning rate & 0.0100 \\
\midrule
Test Accuracy & 0.829723 \\
Test Precision & 0.832371 \\
Test Recall & 0.829723 \\
Test F1-score & 0.831044 \\
\bottomrule
\end{tabular}
\caption{Najlepsza konfiguracja - Manual MLP, Adult Income}
\end{table}

\paragraph{Keras MLP}

\begin{table}[H]
\centering
\begin{tabular}{ll}
\toprule
\textbf{Hiperparametr} & \textbf{Wartość} \\
\midrule
Liczba warstw ukrytych & 3 \\
Liczba neuronów & 16 \\
Learning rate & 0.0050 \\
\midrule
Test Accuracy & 0.827123 \\
Test Precision & 0.829648 \\
Test Recall & 0.827123 \\
Test F1-score & 0.828384 \\
\bottomrule
\end{tabular}
\caption{Najlepsza konfiguracja - Keras MLP, Adult Income}
\end{table}

\subsubsection{Porównanie}

\begin{table}[H]
\centering
\begin{tabular}{lcc}
\toprule
\textbf{Metryka} & \textbf{Manual MLP} & \textbf{Keras MLP} \\
\midrule
Test Accuracy & 0.829723 & 0.827123 \\
Test Precision & 0.832371 & 0.829648 \\
Test Recall & 0.829723 & 0.827123 \\
Test F1-score & 0.831044 & 0.828384 \\
\bottomrule
\end{tabular}
\caption{Porównanie Manual vs Keras - Adult Income}
\end{table}


% =============================================================================
\subsection{Loan Approval (Klasyfikacja binarna)}
% =============================================================================

\subsubsection{Najlepsze konfiguracje}

\paragraph{Manual MLP}

\begin{table}[H]
\centering
\begin{tabular}{ll}
\toprule
\textbf{Hiperparametr} & \textbf{Wartość} \\
\midrule
Liczba warstw ukrytych & 3 \\
Liczba neuronów & 32 \\
Learning rate & 0.0200 \\
\midrule
Test Accuracy & 0.863117 \\
Test Precision & 0.863121 \\
Test Recall & 0.863117 \\
Test F1-score & 0.863119 \\
\bottomrule
\end{tabular}
\caption{Najlepsza konfiguracja - Manual MLP, Loan Approval}
\end{table}

\paragraph{Keras MLP}

\begin{table}[H]
\centering
\begin{tabular}{ll}
\toprule
\textbf{Hiperparametr} & \textbf{Wartość} \\
\midrule
Liczba warstw ukrytych & 2 \\
Liczba neuronów & 64 \\
Learning rate & 0.0200 \\
\midrule
Test Accuracy & 0.862969 \\
Test Precision & 0.863355 \\
Test Recall & 0.862971 \\
Test F1-score & 0.863163 \\
\bottomrule
\end{tabular}
\caption{Najlepsza konfiguracja - Keras MLP, Loan Approval}
\end{table}

\subsubsection{Porównanie}

\begin{table}[H]
\centering
\begin{tabular}{lcc}
\toprule
\textbf{Metryka} & \textbf{Manual MLP} & \textbf{Keras MLP} \\
\midrule
Test Accuracy & 0.863117 & 0.862969 \\
Test Precision & 0.863121 & 0.863355 \\
Test Recall & 0.863117 & 0.862971 \\
Test F1-score & 0.863119 & 0.863163 \\
\bottomrule
\end{tabular}
\caption{Porównanie Manual vs Keras - Loan Approval}
\end{table}


% =============================================================================
\subsection{Stock Market (Regresja)}
% =============================================================================

\subsubsection{Najlepsze konfiguracje}

\paragraph{Manual MLP}

\begin{table}[H]
\centering
\begin{tabular}{ll}
\toprule
\textbf{Hiperparametr} & \textbf{Wartość} \\
\midrule
Liczba warstw ukrytych & 2 \\
Liczba neuronów & 8 \\
Learning rate & 0.0200 \\
\midrule
Test MSE & 149.155077 \\
Test MAE & 9.931640 \\
Test R² & -1.248412 \\
\bottomrule
\end{tabular}
\caption{Najlepsza konfiguracja - Manual MLP, Stock Market}
\end{table}

\paragraph{Keras MLP}

\begin{table}[H]
\centering
\begin{tabular}{ll}
\toprule
\textbf{Hiperparametr} & \textbf{Wartość} \\
\midrule
Liczba warstw ukrytych & 1 \\
Liczba neuronów & 64 \\
Learning rate & 0.0010 \\
\midrule
Test MSE & 0.076797 \\
Test MAE & 0.204588 \\
Test R² & 0.998842 \\
\bottomrule
\end{tabular}
\caption{Najlepsza konfiguracja - Keras MLP, Stock Market}
\end{table}

\paragraph{Keras CNN 1D}

\begin{table}[H]
\centering
\begin{tabular}{ll}
\toprule
\textbf{Hiperparametr} & \textbf{Wartość} \\
\midrule
Liczba warstw Conv1D & 1 \\
Liczba filtrów & 64 \\
Learning rate & 0.0010 \\
Optimizer & rmsprop \\
\midrule
Test MSE & 0.102108 \\
Test MAE & 0.248630 \\
Test R² & 0.998461 \\
\bottomrule
\end{tabular}
\caption{Najlepsza konfiguracja - CNN 1D, Stock Market}
\end{table}

\paragraph{Keras LSTM}

\begin{table}[H]
\centering
\begin{tabular}{ll}
\toprule
\textbf{Hiperparametr} & \textbf{Wartość} \\
\midrule
Liczba warstw LSTM & 1 \\
Liczba jednostek LSTM & 32 \\
Learning rate & 0.0010 \\
Optimizer & rmsprop \\
\midrule
Test MSE & 0.184879 \\
Test MAE & 0.291519 \\
Test R² & 0.997213 \\
\bottomrule
\end{tabular}
\caption{Najlepsza konfiguracja - LSTM, Stock Market}
\end{table}

\subsubsection{Porównanie}

\begin{table}[H]
\centering
\begin{tabular}{lcccc}
\toprule
\textbf{Metryka} & \textbf{Manual MLP} & \textbf{Keras MLP} & \textbf{CNN 1D} & \textbf{LSTM} \\
\midrule
Test MSE & 149.155077 & 0.076797 & 0.102108 & 0.184879 \\
Test MAE & 9.931640 & 0.204588 & 0.248630 & 0.291519 \\
Test R² & -1.248412 & 0.998842 & 0.998461 & 0.997213 \\
\bottomrule
\end{tabular}
\caption{Porównanie wszystkich modeli - Stock Market}
\end{table}

\textbf{Uwaga:} Manual MLP uzyskał bardzo słabe wyniki na tym zbiorze (ujemny R²), co wskazuje na problemy z konwergencją. Keras MLP osiągnął najlepsze wyniki, wyprzedzając nawet zaawansowane architektury CNN 1D i LSTM.


% =============================================================================
\subsection{Student Performance (Regresja)}
% =============================================================================

\subsubsection{Najlepsze konfiguracje}

\paragraph{Manual MLP}

\begin{table}[H]
\centering
\begin{tabular}{ll}
\toprule
\textbf{Hiperparametr} & \textbf{Wartość} \\
\midrule
Liczba warstw ukrytych & 1 \\
Liczba neuronów & 8 \\
Learning rate & 0.0010 \\
\midrule
Test MSE & 5.842659 \\
Test MAE & 1.169886 \\
Test R² & 0.646251 \\
\bottomrule
\end{tabular}
\caption{Najlepsza konfiguracja - Manual MLP, Student Performance}
\end{table}

\paragraph{Keras MLP}

\begin{table}[H]
\centering
\begin{tabular}{ll}
\toprule
\textbf{Hiperparametr} & \textbf{Wartość} \\
\midrule
Liczba warstw ukrytych & 1 \\
Liczba neuronów & 32 \\
Learning rate & 0.0200 \\
\midrule
Test MSE & 5.788280 \\
Test MAE & 1.145609 \\
Test R² & 0.649544 \\
\bottomrule
\end{tabular}
\caption{Najlepsza konfiguracja - Keras MLP, Student Performance}
\end{table}

\subsubsection{Porównanie}

\begin{table}[H]
\centering
\begin{tabular}{lcc}
\toprule
\textbf{Metryka} & \textbf{Manual MLP} & \textbf{Keras MLP} \\
\midrule
Test MSE & 5.842659 & 5.788280 \\
Test MAE & 1.169886 & 1.145609 \\
Test R² & 0.646251 & 0.649544 \\
\bottomrule
\end{tabular}
\caption{Porównanie Manual vs Keras - Student Performance}
\end{table}


% =============================================================================
\subsection{Fashion MNIST (Klasyfikacja obrazów)}
% =============================================================================

\subsubsection{Najlepsze konfiguracje}

\paragraph{Manual MLP}

\begin{table}[H]
\centering
\begin{tabular}{ll}
\toprule
\textbf{Hiperparametr} & \textbf{Wartość} \\
\midrule
Liczba warstw ukrytych & 2 \\
Liczba neuronów & 128 \\
Learning rate & 0.0010 \\
\midrule
Test Accuracy & 0.740300 \\
Test Precision & 0.741498 \\
Test Recall & 0.740300 \\
Test F1-score & 0.740899 \\
\bottomrule
\end{tabular}
\caption{Najlepsza konfiguracja - Manual MLP, Fashion MNIST}
\end{table}

\paragraph{Keras MLP}

\begin{table}[H]
\centering
\begin{tabular}{ll}
\toprule
\textbf{Hiperparametr} & \textbf{Wartość} \\
\midrule
Liczba warstw ukrytych & 3 \\
Liczba neuronów & 128 \\
Learning rate & 0.0010 \\
Optimizer & sgd \\
\midrule
Test Accuracy & 0.732200 \\
Test Precision & 0.735203 \\
Test Recall & 0.732200 \\
Test F1-score & 0.733698 \\
\bottomrule
\end{tabular}
\caption{Najlepsza konfiguracja - Keras MLP, Fashion MNIST}
\end{table}

\paragraph{Keras CNN}

\begin{table}[H]
\centering
\begin{tabular}{ll}
\toprule
\textbf{Hiperparametr} & \textbf{Wartość} \\
\midrule
Liczba warstw Conv2D & 2 \\
Filtry & [64, 128] \\
Learning rate & 0.0010 \\
Optimizer & adam \\
\midrule
Test Accuracy & 0.928800 \\
Test Precision & 0.928457 \\
Test Recall & 0.928800 \\
Test F1-score & 0.928628 \\
\bottomrule
\end{tabular}
\caption{Najlepsza konfiguracja - Keras CNN, Fashion MNIST}
\end{table}

\subsubsection{Porównanie}

\begin{table}[H]
\centering
\begin{tabular}{lccc}
\toprule
\textbf{Metryka} & \textbf{Manual MLP} & \textbf{Keras MLP} & \textbf{Keras CNN} \\
\midrule
Test Accuracy & 0.740300 & 0.732200 & \textbf{0.928800} \\
Test Precision & 0.741498 & 0.735203 & \textbf{0.928457} \\
Test Recall & 0.740300 & 0.732200 & \textbf{0.928800} \\
Test F1-score & 0.740899 & 0.733698 & \textbf{0.928628} \\
\bottomrule
\end{tabular}
\caption{Porównanie wszystkich modeli - Fashion MNIST}
\end{table}

\textbf{Wnioski:}
\begin{itemize}
    \item CNN osiąga znacząco lepsze wyniki (92.88\% accuracy) niż MLP (74.03\% / 73.22\%)
    \item Przewaga CNN wynika z wykorzystania lokalnych wzorców przestrzennych w obrazach
    \item MLP musi traktować obraz jako płaski wektor 784 cech, tracąc informację o strukturze 2D
\end{itemize}


% =============================================================================
\subsection{Podsumowanie wyników}
% =============================================================================

\begin{table}[H]
\centering
\begin{tabular}{lcccc}
\toprule
\textbf{Zbiór danych} & \textbf{Typ} & \textbf{Manual MLP} & \textbf{Keras MLP} & \textbf{Najlepszy model} \\
\midrule
Adult Income & Klasyfikacja & 82.97\% & 82.71\% & Manual MLP \\
Loan Approval & Klasyfikacja & 86.31\% & 86.30\% & Manual MLP \\
Stock Market & Regresja & $R^2$ = -1.25 & $R^2$ = 0.999 & Keras MLP \\
Student Perf. & Regresja & $R^2$ = 0.646 & $R^2$ = 0.650 & Keras MLP \\
Fashion MNIST & Klasyfikacja & 74.03\% & 73.22\% & CNN (92.88\%) \\
\bottomrule
\end{tabular}
\caption{Podsumowanie najlepszych wyników dla wszystkich zbiorów danych}
\end{table}

\textbf{Główne obserwacje:}
\begin{enumerate}
    \item \textbf{Klasyfikacja tabularna:} Manual MLP i Keras MLP osiągają porównywalne wyniki, różnica $<$ 0.5\%
    \item \textbf{Regresja:} Keras znacząco przewyższa Manual MLP, szczególnie dla Stock Market
    \item \textbf{Obrazy:} CNN zdecydowanie lepszy od MLP dla Fashion MNIST (+18.85\% accuracy)
    \item \textbf{Wpływ optymalizatora:} Adam i RMSprop generalnie lepsze od SGD dla skomplikowanych zadań
\end{enumerate}



% ============================================================================
\section{Wnioski}
% ============================================================================

\subsection{Porównanie implementacji}

Na podstawie przeprowadzonych eksperymentów można wyciągnąć następujące wnioski:

\textbf{Implementacja ręczna (Manual MLP):}
\begin{itemize}
    \item Pełna kontrola nad procesem treningu i możliwość debugowania
    \item Wysoka wartość edukacyjna -- zrozumienie backpropagation od podstaw
    \item Wolniejsza ze względu na brak optymalizacji niskopoziomowych
    \item Dobre wyniki dla klasyfikacji tabelarycznej (82.97\% Adult Income, 86.31\% Loan Approval)
    \item Problemy z konwergencją dla skomplikowanych danych regresyjnych (ujemny R² dla Stock Market)
\end{itemize}

\textbf{Implementacja Keras:}
\begin{itemize}
    \item Znacznie szybsza (5-10x) dzięki zoptymalizowanemu backendowi
    \item Łatwość implementacji różnych architektur (MLP, CNN, LSTM)
    \item Porównywalne wyniki dla klasyfikacji (różnica <0.5\%)
    \item Zdecydowanie lepsze dla regresji (R² = 0.999 vs -1.25 dla Stock Market)
    \item Możliwość wykorzystania różnych optimizerów (Adam, RMSprop)
\end{itemize}

\subsection{Wyniki eksperymentów}

\textbf{Klasyfikacja:}
\begin{itemize}
    \item Obie implementacje osiągnęły porównywalne wyniki (różnica <0.5\%)
    \item Adult Income: Manual 82.97\% vs Keras 82.71\%
    \item Loan Approval: Manual 86.31\% vs Keras 86.30\%
    \item Wyniki zgodne z literaturą dla tych zbiorów danych
\end{itemize}

\textbf{Regresja:}
\begin{itemize}
    \item Keras znacząco lepszy od Manual MLP
    \item Stock Market: Keras R²=0.999 vs Manual R²=-1.25
    \item Student Performance: Keras R²=0.650 vs Manual R²=0.646
    \item Manual MLP ma problemy z konwergencją dla danych finansowych
\end{itemize}

\textbf{Fashion MNIST (obrazy):}
\begin{itemize}
    \item CNN zdecydowanie przewyższa MLP (+18.85\% accuracy)
    \item Keras CNN: 92.88\% vs Manual MLP: 74.03\% vs Keras MLP: 73.22\%
    \item Potwierdza to, że CNN lepiej nadaje się do analizy obrazów
\end{itemize}

\subsection{Problemy napotkane}

\begin{enumerate}
    \item \textbf{Konwergencja Manual MLP dla regresji:} Model ręczny miał problemy z optymalizacją dla danych Stock Market. Prawdopodobna przyczyna: zbyt prosty optymalizator (SGD bez momentum).

    \item \textbf{Wymiary warstw w CNN 1D:} Przy małej liczbie cech (5 dla Stock Market) pooling zbyt mocno redukował wymiary. Rozwiązanie: warunkowe stosowanie poolingu.

    \item \textbf{Czas treningu:} Eksperymenty na Fashion MNIST zajmowały kilka godzin. Rozwiązanie: zwiększenie batch size, zmniejszenie liczby epok.

    \item \textbf{Zarządzanie hiperparametrami:} Grid search generował setki kombinacji. Rozwiązanie: systematyczne śledzenie eksperymentów w plikach Excel.
\end{enumerate}

\subsection{Możliwe usprawnienia}

\begin{itemize}
    \item \textbf{Zaimplementowane:}
    \begin{itemize}
        \item Momentum w Manual MLP (poprawia konwergencję)
        \item Różne optymalizatory w Keras (Adam, RMSprop, SGD)
        \item CNN dla obrazów i CNN 1D dla szeregów czasowych
        \item LSTM dla danych sekwencyjnych
    \end{itemize}

    \item \textbf{Do zaimplementowania:}
    \begin{itemize}
        \item Regularyzacja (L1, L2, Dropout) -- zapobiega overfittingowi
        \item Early stopping -- automatyczne zatrzymanie treningu
        \item Learning rate decay -- zmniejszanie LR w trakcie treningu
        \item Batch normalization -- stabilizuje trening
        \item Data augmentation (dla obrazów) -- zwiększa ilość danych treningowych
    \end{itemize}
\end{itemize}

% ============================================================================
\section{Bibliografia}
% ============================================================================

\begin{enumerate}
    \item Goodfellow, I., Bengio, Y., Courville, A. (2016). \textit{Deep Learning}. MIT Press.
    \item Nielsen, M. A. (2015). \textit{Neural Networks and Deep Learning}. Determination Press.
    \item Géron, A. (2019). \textit{Hands-On Machine Learning with Scikit-Learn, Keras, and TensorFlow}. O'Reilly Media.
    \item TensorFlow Documentation: \url{https://www.tensorflow.org/}
    \item Keras Documentation: \url{https://keras.io/}
    \item UCI Machine Learning Repository: \url{https://archive.ics.uci.edu/ml/}
\end{enumerate}

% ============================================================================
\appendix
\section{Kod źródłowy}
% ============================================================================

Pełny kod źródłowy projektu dostępny jest w repozytorium GitHub:

\url{https://github.com/Sornat11/NeuralNetwork}

Struktura projektu:
\begin{itemize}
    \item \texttt{src/manual\_mlp/} -- Implementacja ręczna
    \item \texttt{src/models/} -- Implementacja Keras
    \item \texttt{data/} -- Zbiory danych i preprocessing
    \item \texttt{utils/} -- Narzędzia pomocnicze (experiment runner, visualization)
    \item \texttt{results/} -- Wyniki eksperymentów i wizualizacje
\end{itemize}

\end{document}
